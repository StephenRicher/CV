\documentclass{resume} % Use the custom resume.cls style

\usepackage[left=0.4 in,top=0.4in,right=0.4 in,bottom=0.4in]{geometry} % Document margins
\newcommand{\tab}[1]{\hspace{.2667\textwidth}\rlap{#1}} 
\newcommand{\itab}[1]{\hspace{0em}\rlap{#1}}
\name{Stephen Richer} % Your name
% You can merge both of these into a single line, if you do not have a website.
\address{+44(7513) 065 917 \\ London, U.K.} 
\address{\href{mailto:stephen.richer@proton.me}{stephen.richer@proton.me} \\ \href{https://www.linkedin.com/in/stephenricher/}{linkedin.com/in/stephenricher} \\ \href{https://github.com/StephenRicher}{github.com/StephenRicher}}  %

\begin{document}

%----------------------------------------------------------------------------------------
%	OBJECTIVE
%----------------------------------------------------------------------------------------

\begin{rSection}{OBJECTIVE}

{Python programmer and data scientist currently working with the NHS to build best practise analytical workflows and machine learning models to address health inequality. Seeking a full-time position in software engineering.}

%----------------------------------------------------------------------------------------
% TECHINICAL STRENGTHS	
%----------------------------------------------------------------------------------------
\begin{rSection}{SKILLS}

\begin{tabular}{ @{} >{\bfseries}l @{\hspace{6ex}} l }
Programming Languages & Python (NumPy, Pandas, Scikit-learn), SQL, Git, Bash, C++, Java
\\
Data Science & Supervised Machine Learning (Decision Trees, Regression), Tableu
\\
Cloud Platforms & Microsoft Azure
\\
Soft Skills & Communication, Critical Thinking, Problem Solving, Adaptability
\\
\end{tabular}\\
\end{rSection}


\end{rSection}

\begin{rSection}{EXPERIENCE}

\textbf{Data Scientist Intern} \hfill Jun 2022 - Present \\
NHS England \hfill \textit{London, U.K}
 \begin{itemize}
    \itemsep -3pt {} 
    \item Built and deployed an AutoML pipeline, implementing CatBoost and Logistic Regression, for predicting healthcare appointment non-attendance at regional NHS trusts. Available at: \href{https://github.com/nhsx/dna-risk-predict}{github.com/nhsx/dna-risk-predict}
    \item Developed ETL data engineering pipelines for processing and aggregating healthcare and demographic data from multiple public source.
    \item Collaborated with NHS stakeholders and analysts to design and implement reproducible analytical workflows for studying healthcare inequalities.
    \item Utilised a variety of statistical approaches and hypothesis tests, including permutation testing and partial correlation, to robustly analyse complex and biased real-world healthcare data with numerous confounding variables.
 \end{itemize}

\textbf{Doctoral Researcher, Biology \& Mathematics} \hfill Oct 2018 - Jun 2022 \\
University of Bath \hfill \textit{Bath, U.K}
 \begin{itemize}
    \itemsep -3pt {} 
    \item Conducted independent research and developed novel computational tools to understand how the structure of DNA impacts cellular function.
    \item Developed HiCFlow, a user-friendly analytical workflow for conducting robust and reproducible bioinformatics analysis. Available at: \href{https://github.com/StephenRicher/HiCFlow}{github.com/StephenRicher/HiCFlow}
    \item Supervised final-year undergraduate students completing projects in computational biology.
    \item Presented research at national and international conferences.
    \item Awarded best final-year PhD presentation at the Bath Departmental Research Day.
 \end{itemize}

\textbf{Research Software Skills Trainer} \hfill Apr 2019 - Jun 2022 \\
University of Bath, Doctoral College \hfill \textit{Bath, U.K}
 \begin{itemize}
    \itemsep -3pt {} 
    \item Developed and delivered materials to train doctoral students in research software engineering best practises.
    \item Curriculum: Python programming and style (PEP8), version control, testing and continuous integration. Available at: \href{https://github.com/Research-Software-Skills-Bath}{github.com/Research-Software-Skills-Bath}
 \end{itemize}

\textbf{Technical Support (IT Services)} \hfill Oct 2017 - Sep 2018 \\
University of Manchester, Hornet \hfill \textit{Manchester, U.K}
 \begin{itemize}
    \itemsep -3pt {} 
    \item Provided IT support, education and security guidance for residential students.
 \end{itemize}

\textbf{Study Coordinator} \hfill Oct 2016 - Jul 2017 \\
MAC Clinical Research \hfill \textit{Leeds, U.K}
 \begin{itemize}
    \itemsep -3pt {} 
    \item Oversaw a phase 3 clinical trial investigating a novel treatment for Alzheimer’s disease.
 \end{itemize}

\textbf{Clinical Trials Assistant} \hfill Oct 2015 - Oct 2016 \\
MAC Clinical Research \hfill \textit{Leeds, U.K}

\end{rSection} 

%----------------------------------------------------------------------------------------
%	EDUCATION SECTION
%----------------------------------------------------------------------------------------

\begin{rSection}{Education}

{\bf PhD, Biology \& Mathematics}, University of Bath \hfill {Expected 2022} \\
Mathematical and bioinformatics-based tools to explore the impact of gene editing on the geometric principles governing the 3D structure of the genome.

{\bf MSc, Bioinformatics \& Systems Biology (Distinction)}, University of Manchester \hfill {2017 - 2018}

{\bf BSc, Biology (First)}, University of Bath \hfill {2011 - 2015}

\end{rSection}

%----------------------------------------------------------------------------------------
%	WORK EXPERIENCE SECTION
%----------------------------------------------------------------------------------------

\begin{rSection}{PROJECTS}
\vspace{-1.25em}
\item \textbf{Real World Data Validation with \textit{validatum}} {Lightweight, user-friendly Python module to automatically detect and flag suspicious data inconsistencies and common data entry errors within real world data. \\
Available at: \href{https://pypi.org/project/validatum/}{pypi.org/project/validatum}}
\item \textbf{Secure Data Encryption with \textit{datasafe}.} {Python based command-line utility for encryption of text files and encryption of Pandas DataFrames while preserving datatype. \\
Available at: \href{https://pypi.org/project/datasafe/}{pypi.org/project/datasafe}}
\item \textbf{Data Science Best Practises - The Titanic Dataset.} {Wrote a popular Kaggle notebook, describing best practise approaches for performing supervised tabular classification using Scikit-learn. \\
Available at: \href{https://www.kaggle.com/code/stever4/titanic-top-5-good-data-science-practices}{Kaggle - Titanic Data Science}}
\end{rSection} 

%----------------------------------------------------------------------------------------

\nocite{*}
\end{document}

